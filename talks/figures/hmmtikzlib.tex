\usepackage{pgfplots}
\usepackage{helvet}
\usepackage[eulergreek]{sansmath}
\usepackage{stringstrings}
\usetikzlibrary{calc}
\usetikzlibrary{fit}
\usetikzlibrary{arrows}
\usetikzlibrary{positioning}
\usetikzlibrary{decorations.markings}
\usetikzlibrary{shapes.geometric}
\usetikzlibrary{shapes.multipart}

\pgfplotsset{
  tick label style = {font=\sansmath\sffamily}
}

\def\stateSize{0.4}
\def\stateHeight{4}
\def\stateArcHeight{0.6}
\def\rectWidth{0.3}
\def\rectHeight{0.15}
\def\rectOffset{-1.2}
\def\featNamePos{-0.7}
\def\stateSep{1.5}

\newcommand{\mystate}[5]{
  \draw[] ($(#1,\stateHeight)+(0,#3)$) circle (\stateSize)
  node[fitting node] (#2) {};
  %% cannot directly compare floating numbers
  \newdimen\dummy
  \dummy = #4 pt
  \ifthenelse{\dummy > 0 pt}{
    \node[] (p) at ($(#2.north)+(0,\stateArcHeight)$) {#4};
    \draw [->, thick] (#2.north) to[out=15,in=0] ($(p.south)-(0,0.05)$)
    to[out=180,in=165] (#2.north);
  }{}
  \dummy = #5 pt
  \ifthenelse{\dummy > 0 pt}{
    \draw [->, thick] ($(#2.south west)+(-0.3,-0.1)$) -- (#2) node [midway, above,
      sloped] {\scalebox{0.7}{#5}};
  }{}
}

\makeatletter
\tikzset{
  fitting node/.style={
    inner sep=0pt,
    fill=none,
    draw=none,
    reset transform,
    fit={(\pgf@pathminx,\pgf@pathminy) (\pgf@pathmaxx,\pgf@pathmaxy)}
  },
  reset transform/.code={\pgftransformreset}
}
\makeatother

\newcommand{\tikzfig}[2]{
  \begin{tikzpicture}[#1] #2 \end{tikzpicture}}

\newcommand{\tscope}[2]{\begin{scope}[x={(#1.south east)},y={(#1.north west)}]
    #2 \end{scope}}

\newcommand{\insertImage}[3]{
  \node[anchor=south west,inner sep=0] (#3) at (0,0)
       {\includegraphics[width=#2\textwidth]
         {#1}};}

\newcommand{\variable}[4]{
    \draw[thick] (#1,#2) circle (\stateSize) node[fitting node] (#3) {};
    \node[#4] (name-#3) at ([yshift=-16] #3) {\textbf{#3}};
}

\newcommand{\constraint}[6]{
  \node (c) at ($(#1,#2)$)
        {\includegraphics[width=0.4\textwidth]{#6}};
  \draw[thick] (#3) -- (c);
  \draw[thick] (#4) -- (c);
  \draw[thick] (#5) -- (c);
}

% #1-#5 node lable #6 node name #7,#8 x,y
\newcommand{\latticeNodeFive}[8]{
  \node[draw, minimum width=3cm, minimum height=1cm, anchor=south] (#6) at (#7, #8)
       {$\substack{#4,#5\\#1,#2,#3}$};
}

% #1-#3 node lable #4 node name #5,#6 x,y
\newcommand{\latticeNodeThree}[6]{
  \node[draw, minimum width=2.4cm, minimum height=0.85cm, anchor=south] (#4) at (#5, #6)
       {#1, #2, #3};
}

% #1,#2 node lable #3 node name #4,#5 x,y
\newcommand{\latticeNodeTwo}[5]{
  \node[draw, minimum width=1.7cm, minimum height=0.85cm, anchor=south] (#3) at (#4, #5)
       {#1, #2};
}

% #1 node lable #2 node name #3,#4 x,y #5 draw border
\newcommand{\latticeNode}[5]{
  \node[#5,minimum width=1.7cm, minimum height=1.06cm, anchor=south] (#2) at (#3,#4) {#1};
}

% the two types of lattices are scaled differently, so need two functions
% #1 supernode label, #2 node block colour #3 node block label
% #4 link block colour #5 link block label, #6 lattice label
\newcommand{\latticeOverlayT}[6]{
  \fill[#2,opacity=0.25]
  let
  \p1=($(#1.west)!0.125!(#1.east)$),
  \p2=($(#1.west)!0.235!(#1.east)$),
  \p3=(#1.south),
  \p4=($(#1.north)!0.08!(#1.south)$)
  in (\x1,\y3) rectangle (\x2, \y4);
  \fill[#4,opacity=0.25]
  let
  \p1=($(#1.west)!0.235!(#1.east)$),
  \p2=($(#1.west)!0.343!(#1.east)$),
  \p3=($(#1.south)!0.057!(#1.north)$),
  \p4=($(#1.north)!0.138!(#1.south)$)
  in (\x1,\y3) rectangle (\x2, \y4);
  \path
  let
  \p1=($(#1.west)!0.1795!(#1.east)$),
  \p2=($(#1.south)$) in node[#2,anchor=south] at ($(\x1,\y2)-(0,0.035)$) {#3};
  \path let \p1=($(#1.west)!0.289!(#1.east)$), \p2=($(#1.south)$) in
  node[#4,anchor=south] at ($(\x1,\y2)-(0,0.035)$) {#5};
  \path let \p1=($(#1.west)!0.5!(#1.east)$), \p2=($(#1.south)$) in node (#1l) at
  ($(\x1,\y2)-(0,0.05)$) {#6};
}

\newcommand{\latticeOverlayW}[6]{
  \fill[#2,opacity=0.25]
  let
  \p1=($(#1.west)!0.177!(#1.east)$),
  \p2=($(#1.west)!0.291!(#1.east)$),
  \p3=(#1.south),
  \p4=($(#1.north)!0.11!(#1.south)$)
  in (\x1,\y3) rectangle (\x2, \y4);
  \fill[#4,opacity=0.25]
  let
  \p1=($(#1.west)!0.291!(#1.east)$),
  \p2=($(#1.west)!0.379!(#1.east)$),
  \p3=($(#1.south)!0.068!(#1.north)$),
  \p4=($(#1.north)!0.171!(#1.south)$)
  in (\x1,\y3) rectangle (\x2, \y4);
  \path
  let
  \p1=($(#1.west)!0.234!(#1.east)$),
  \p2=($(#1.south)$) in node[#2,anchor=south] at ($(\x1,\y2)-(0,0.035)$) {#3};
  \path let \p1=($(#1.west)!0.335!(#1.east)$), \p2=($(#1.south)$) in
  node[#4,anchor=south] at ($(\x1,\y2)-(0,0.035)$) {#5};
  \path let \p1=($(#1.west)!0.5!(#1.east)$), \p2=($(#1.south)$) in node (#1l) at
  ($(\x1,\y2)-(0,0.05)$) {#6};
}

\tikzstyle{vecArrow} = [thick, decoration={markings,mark=at position
   1 with {\arrow[semithick]{open triangle 60}}},
   double distance=3pt, shorten >= 5.5pt,
   preaction = {decorate},
   postaction = {draw,line width=1.4pt, white,shorten >= 4.5pt}]
\tikzstyle{innerWhite} = [semithick, white,line width=1.4pt, shorten >= 4.5pt]
