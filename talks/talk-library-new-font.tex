\documentclass[usenames,dvipsnames,svgnames,table,unknownkeysallowed,aspectratio=169]{beamer}
\usepackage{etex}
\usepackage{times}
% \usepackage{graphicx}
\usepackage{multimedia}
\usepackage{amsmath}
\usepackage{tikz}
\usepackage{array}
\usepackage{hyperref}
\usepackage[all]{xy}
\usepackage{multirow}
\usepackage{cancel}
\usepackage[normalem]{ulem}
\usepackage{esdiff}
\usepackage{tabularx}
\usepackage{tabulary}
% \usepackage{switch}
% \usepackage{tikz-3dplot}
\usepackage{etoolbox}
\usepackage{mathtools}
\usepackage{wrapfig}
\usepackage{pgf}
\usepackage{paralist}
\usepackage{cancel}
\usepackage{enumitem}
\usepackage{makecell}%
\usepackage{pdfpages}
\usepackage[letterspace=0,tracking=true]{microtype}
\usepackage[romanian,russian,english]{babel}
\usepackage[encapsulated]{CJK}
\usepackage[T2A,T1]{fontenc}
\usepackage[utf8]{inputenc}
\usepackage{scalefnt}
\usepackage{qtree}
\usepackage{xspace}
\usepackage[boxed,lined,linesnumbered]{algorithm2e}
\usepackage{ifthen}
\usepackage{xparse}
\usepackage{enumitem}
\usepackage[skins]{tcolorbox}
\usepackage{nicefrac}
\usepackage{graphbox}
\usepackage[super]{nth}
\usepackage{xhfill}
\usepackage{ccg-latex}
\usepackage{linegoal}
\usepackage{adjustbox}

\makeatletter
\newcommand*{\overlaynumber}{\number\beamer@slideinframe}
\makeatother
\newcommand*{\debugoverlay}{\Cite{\overlaynumber}}

\newcommand{\TODO}{{\red TODO}\xspace}

\makeatletter
\newcommand{\nameoverlay}[1]{%
  \@ifundefined{c@#1}{%
    \newcounter{#1}%
    \setcounter{#1}{\overlaynumber}%
  }{}
}
\makeatother

\NewDocumentCommand\placeholder{mmo}{
  \framebox{
    \begin{minipage}[c][#2]{#1}
      \centering
      \IfNoValueTF{#3}{Placeholder}{#3}
    \end{minipage}
  }
}

\usepackage[absolute,overlay]{textpos}
\newcommand\Cite[1]{%
  \begin{textblock*}{\paperwidth}(0pt,0.98\textheight)
    \begin{footnotesize}
      \hfill #1
    \end{footnotesize}
  \end{textblock*}}

\newcommand\Collaborators[1]{%
  \begin{textblock*}{\paperwidth}(0pt,0.8\textheight)
    \begin{footnotesize}
      \hfill #1
    \end{footnotesize}
  \end{textblock*}}

\newcommand\CollaboratorsImage[1]{%
  \begin{textblock*}{\paperwidth}(0pt,0.75\textheight)
    % \begin{footnotesize}
    %   \hfill #1
    % \end{footnotesize}
    \begin{scriptsize}
      \hfill #1
    \end{scriptsize}
  \end{textblock*}}

\newcommand\Video[4]{%
  \only<#1>{\includegraphics[width=#4\textwidth]{videos/#3}}%
  \only<#2>{\movie[externalviewer] {\includegraphics[width=#4\textwidth]{videos/#3}} {videos/#3}}%
}

\newcommand\Image[3]{%
  \only<#1>{\includegraphics[width=#3\textwidth]{#2}}%
}

\newcommand\image[2]{%
  \includegraphics[width=#2\textwidth]{#1}%
}

\newcommand\Task[6]{%
  \uncover<#1->{\vphantom{$\displaystyle\int_{\blue v}$}%
    \lightBlueOn<#2-#3>{#4}&%
    \uncover<#2->{#5}&%
    \uncover<#2->{%
      \begin{tiny}#6\end{tiny}%
    }\\}%
}

%% I like this but there's the small issue that math looks bad and some slides have wrong spacing
\RequirePackage[scale=0.95]{tgpagella}
% \IfFileExists{mathpazo.sty}{\RequirePackage[osf,sc]{mathpazo}}{}
% \IfFileExists{palatino.sty}{\RequirePackage{palatino}}{}
% \IfFileExists{helvet.sty}{\RequirePackage[scaled=0.90]{helvet}}{}
% \IfFileExists{beramono.sty}{\RequirePackage[scaled=0.85]{beramono}}{}
% \RequirePackage{textcomp}

\newcommand{\FIXME}{\textbf{\textcolor{red}{FIXME}}}
\newcommand{\theset}[1]{\{#1\}}
\newcommand\beameritemnestingprefix{}
\DeclareMathOperator*{\argmin}{argmin}
\DeclareMathOperator*{\argmax}{argmax}

\def\boolNeuroscience{1}
\def\boolBasicfMRI{0}
\def\boolVideoEvents{1}
\def\boolAssemblyImitation{0}
\def\boolPublications{0}

\usenavigationsymbolstemplate{}

% Add a period to the end of an abbreviation unless there's one
% already, then \xspace.
\makeatletter
\DeclareRobustCommand\onedot{\futurelet\@let@token\@onedot}
\def\@onedot{\ifx\@let@token.\else.\null\fi\xspace}

\def\etal{\emph{et al}\onedot}
\newcommand{\eg}{e.g.,\xspace}
\newcommand{\Eg}{E.g.,\xspace}
\newcommand{\ie}{i.e.,\xspace}
\newcommand{\Ie}{I.e.,\xspace}
\newcommand{\etc}{etc\onedot}
\newcommand{\vs}{vs\onedot}
\makeatother

% TODO Do I really want these versions?
% \def\etal{\emph{et al}\onedot}
% \newcommand{\eg}{e.g.,}
% \newcommand{\Eg}{E.g.,}
% \newcommand{\ie}{i.e.,}
% \newcommand{\Ie}{I.e.,}
% \newcommand{\etc}{etc\onedot}
% \newcommand{\vs}{vs\onedot}
% \makeatother

\usetikzlibrary{3d}
\usetikzlibrary{arrows, arrows.meta}
\usetikzlibrary{backgrounds}
\usetikzlibrary{bayesnet}
\usetikzlibrary{calc}
\usetikzlibrary{decorations.markings}
\usetikzlibrary{decorations.pathreplacing}
\usetikzlibrary{fit}
\usetikzlibrary{overlay-beamer-styles}
\usetikzlibrary{positioning}
\usetikzlibrary{shapes.geometric}
\usetikzlibrary{shapes.misc}
\usetikzlibrary{shapes.multipart}
\usetikzlibrary{shapes}
\usetikzlibrary{tikzmark}
\usetikzlibrary{calligraphy}

\tikzset{cross/.style={cross out, draw=black, minimum size=2*(#1-\pgflinewidth), inner sep=0pt, outer sep=0pt},
%default radius will be 1pt. 
cross/.default={1pt}}

\newcommand{\savedx}{0}
\newcommand{\savedy}{0}
\newcommand{\savedz}{0}

% http://tex.stackexchange.com/questions/6135/how-to-make-beamer-overlays-with-tikz-node
\tikzset{onslide/.code args={<#1>#2}{%
    \only<#1>{\pgfkeysalso{#2}} % \pgfkeysalso doesn't change the path
  }}
\tikzset{temporal/.code args={<#1>#2#3#4}{%
    \temporal<#1>{\pgfkeysalso{#2}}{\pgfkeysalso{#3}}{\pgfkeysalso{#4}} % \pgfkeysalso doesn't change the path
  }}

\newcommand{\TicTacToe}{\textsc{Tic Tac Toe}}
\newcommand{\Hexapawn}{\textsc{Hexapawn}}
\newcommand{\Prolog}{\mbox{\textsc{Prolog}}}
\newcommand{\Progol}{\mbox{\textsc{Progol}}}
\newcommand{\Scheme}{\mbox{\textsc{Scheme}}}
\newenvironment{lisp}{\begin{tt}\begin{tabular}[t]{l}}{\end{tabular}\end{tt}}
\newcommand{\protagonist}{\mbox{\small{\textbf{protagonist}}}}
\newcommand{\antagonist}{\mbox{\small{\textbf{antagonist}}}}
\newcommand{\wannabe}{\mbox{\small{\textbf{wannabe}}}}
\newcommand{\ttt}{\mbox{\small{\textsc{Tic Tac Toe}}}}
\newcommand{\hexp}{\mbox{\small{\textsc{Hexapawn}}}}
\newcommand{\hexpvar}{\mbox{\small{\textsc{\& $4$ Variants}}}}

\definecolor{green}{rgb}{0,0.69,0}
\definecolor{brightGreen}{rgb}{0,0.9,0}
\definecolor{pink}{RGB}{235,0,235}
\definecolor{blue}{RGB}{0,80,255}
\definecolor{brightBlue}{RGB}{0,0,255}
\definecolor{slateblue}{RGB}{71,60,139}
\definecolor{dullyellow}{RGB}{145,145,0}
\newcommand{\pink}{\color{pink}}
\newcommand{\brightGreen}{\color{brightGreen}}
\newcommand{\purple}{\color{purple}}
\newcommand{\brightBlue}{\color{brightBlue}}
\newcommand{\slateblue}{\color{slateblue}}
\newcommand{\dullyellow}{\color{dullyellow}}
\newcommand{\positiveExample}[1]{{\color{green} #1}}
\newcommand{\negativeExample}[1]{{\color{red} #1}}

\newcommand{\prologBoard}[9]{$\!\left[
    \begin{array}{@{}l@{}l@{}l@{}l@{}l@{}}
      \left[\right.&\makebox[1em][c]{\text{#1},}&\makebox[1em][c]{\text{#2,}}&\makebox[0.5em][c]{\text{#3}}&\left.\right]\\
      \left[\right.&\text{#4,}&\text{#5,}&\text{#6}&\left.\right]\\
      \left[\right.&\text{#7,}&\text{#8,}&\text{#9}&\left.\right]\\
    \end{array}%
  \right]\!$}

\newcommand{\pgffancycircle}[2]{\pgfcircle[stroke]{#1}{#2}\pgfcircle[fill]{#1}{#2}}

\newcommand{\pyramid}%
{%
  % bottom
  \draw[black,very thick, fill=white, fill opacity=0.8] (-1,-1,-1) -- (-1,1,-1) -- (1,1,-1) -- (1,-1,-1) -- (-1, -1, -1);
  \begin{scope}[shift={(0,0,0.2)},scale=0.8]
    \draw[black,very thick, fill=white, fill opacity=0.8] (-1,-1,-1) -- (-1,1,-1) -- (1,1,-1) -- (1,-1,-1) -- (-1, -1, -1);
  \end{scope}
  \begin{scope}[shift={(0,0,0.4)},scale=0.6]
    \draw[black,very thick, fill=white, fill opacity=0.8] (-1,-1,-1) -- (-1,1,-1) -- (1,1,-1) -- (1,-1,-1) -- (-1, -1, -1);
  \end{scope}
  \begin{scope}[shift={(0,0,0.5)},scale=0.4]
    \draw[black,very thick, fill=white, fill opacity=0.8] (-1,-1,-1) -- (-1,1,-1) -- (1,1,-1) -- (1,-1,-1) -- (-1, -1, -1);
  \end{scope}
}

\newcommand{\prism}%
{%
  % bottom
  \draw[black,very thick, fill=white, fill opacity=0.8] (-1,-1,-1) -- (-1,1,-1) -- (1,1,-1) -- (1,-1,-1) -- (-1, -1, -1);
  \begin{scope}[shift={(0,0,0.4)}]
    \draw[black,very thick, fill=white, fill opacity=0.8] (-1,-1,-1) -- (-1,1,-1) -- (1,1,-1) -- (1,-1,-1) -- (-1, -1, -1);
  \end{scope}
  \begin{scope}[shift={(0,0,0.7)}]
    \draw[black,very thick, fill=white, fill opacity=0.8] (-1,-1,-1) -- (-1,1,-1) -- (1,1,-1) -- (1,-1,-1) -- (-1, -1, -1);
  \end{scope}
  \begin{scope}[shift={(0,0,1.0)}]
    \draw[black,very thick, fill=white, fill opacity=0.8] (-1,-1,-1) -- (-1,1,-1) -- (1,1,-1) -- (1,-1,-1) -- (-1, -1, -1);
  \end{scope}
}

\newcommand{\verticalDiagram}[1]{%
  \draw[thick] (0,0) rectangle node (events) {Event recognizer} (3,0.5);
  \draw[thick] (0,-0.5) rectangle node (tracks) {Tracker} (3,-1);
  \draw[thick] (0,-1.5) rectangle node (objects) {Object detector} (3,-2);
  \draw[->,thick] (objects) -- (tracks);
  \draw[->,thick] (tracks) -- (events);
  \only<#1>{%
    \draw[red,->,thick] ($(events)+(1ex,-0.3)$) -- ($(tracks)+(1ex,0.3)$);
    \draw[red,->,thick] ($(tracks)+(1ex,-0.3)$) -- ($(objects)+(1ex,0.3)$);}
}

\newcommand{\rotateRPY}[4][0/0/0]% point to be saved to \savedxyz, roll, pitch, yaw
{   \pgfmathsetmacro{\rollangle}{#2}
  \pgfmathsetmacro{\pitchangle}{#3}
  \pgfmathsetmacro{\yawangle}{#4}

  % to what vector is the x unit vector transformed, and which 2D vector is this?
  \pgfmathsetmacro{\newxx}{cos(\yawangle)*cos(\pitchangle)}% a
  \pgfmathsetmacro{\newxy}{sin(\yawangle)*cos(\pitchangle)}% d
  \pgfmathsetmacro{\newxz}{-sin(\pitchangle)}% g
  \path (\newxx,\newxy,\newxz);
  \pgfgetlastxy{\nxx}{\nxy};

  % to what vector is the y unit vector transformed, and which 2D vector is this?
  \pgfmathsetmacro{\newyx}{cos(\yawangle)*sin(\pitchangle)*sin(\rollangle)-sin(\yawangle)*cos(\rollangle)}% b
  \pgfmathsetmacro{\newyy}{sin(\yawangle)*sin(\pitchangle)*sin(\rollangle)+ cos(\yawangle)*cos(\rollangle)}% e
  \pgfmathsetmacro{\newyz}{cos(\pitchangle)*sin(\rollangle)}% h
  \path (\newyx,\newyy,\newyz);
  \pgfgetlastxy{\nyx}{\nyy};

  % to what vector is the z unit vector transformed, and which 2D vector is this?
  \pgfmathsetmacro{\newzx}{cos(\yawangle)*sin(\pitchangle)*cos(\rollangle)+ sin(\yawangle)*sin(\rollangle)}
  \pgfmathsetmacro{\newzy}{sin(\yawangle)*sin(\pitchangle)*cos(\rollangle)-cos(\yawangle)*sin(\rollangle)}
  \pgfmathsetmacro{\newzz}{cos(\pitchangle)*cos(\rollangle)}
  \path (\newzx,\newzy,\newzz);
  \pgfgetlastxy{\nzx}{\nzy};

  % transform the point given by #1
  \foreach \x/\y/\z in {#1}
  {   \pgfmathsetmacro{\transformedx}{\x*\newxx+\y*\newyx+\z*\newzx}
    \pgfmathsetmacro{\transformedy}{\x*\newxy+\y*\newyy+\z*\newzy}
    \pgfmathsetmacro{\transformedz}{\x*\newxz+\y*\newyz+\z*\newzz}
    \xdef\savedx{\transformedx}
    \xdef\savedy{\transformedy}
    \xdef\savedz{\transformedz}
  }
}

\newcommand{\mystateX}[4]{%
  \draw[] (#1,\stateHeight) circle (\stateSize) node[fitting node] (#2) {};
  \draw [->] (#2.north) to[out=45,in=0] ($(#2.north)+(0,\stateArcHeight)$) to[out=180,in=135] (#2.north);
  \uncover<#4>{%
    \draw[] ($(#2.south)+(0,\rectOffset)-(\rectSize,\rectSize)$) rectangle
    ($(#2.south)+(0,\rectOffset)+(\rectSize,\rectSize)$) node[fitting node] (#3) {};
    \draw [->] (#2.south) -- (#3.north);}
}

\newcommand{\myframe}[4]{%
  \draw[] ($(#1,\frameYOffset)-(\frameWidth,\frameHeight)$) rectangle
  ($(#1,\frameYOffset)+(\frameWidth,\frameHeight)$) node[fitting node] (#2) {};
  \uncover<#4>{\draw[] ($(#2.north)+(0,\frameYOffset)-(\rectSize,\rectSize)$) rectangle
    ($(#2.north)+(0,\frameYOffset)+(\rectSize,\rectSize)$) node[fitting node] (#3) {};
    \draw [->] (#2.north) -- (#3.south);}
}

\makeatletter
\newcommand\myconnect[2][\\]{%
  \global\def\my@node{#1}%
  \my@connect #2,\relax\noexpand\@eolst}

\def\my@connect #1,#2\@eolst{%
  \draw [->] (\my@node.south) -- (#1.north);
  \ifx\relax#2\relax
  #1
  \else
  \my@connect #2\@eolst%
  \fi}
\makeatother

\newcommand{\systemDiagram}[8]{%
  % back arrows or decisions #1, object detector decision #2, tracker decision #3
  \begin{figure}
    \begin{tikzpicture}
%      \uncover<#1>{\draw[thick] (0,0) rectangle node[fitting node] (objects) {\\\vspace{-1ex}Objects} (2,1);}
      \uncover<#2>{\draw[thick] (3,0) rectangle node[fitting node] (tracks) {\\\vspace{-1ex}Tracks} (5,1);}
      \uncover<#3>{\draw[thick] (6,0) rectangle node[fitting node] (events) {\\\vspace{-1ex}Events} (8,1);}
      \uncover<#4>{\draw[thick] (9,0) rectangle node[fitting node] (sentences) {\\\vspace{-1ex}Sentences} (11,1);}
 %     \uncover<#2>{\draw[->,thick] (objects) -- (tracks);}
      \uncover<#3>{\draw[->,thick] (tracks) -- (events);}
      \uncover<#4>{\draw[->,thick] (events) -- (sentences);}
      \uncover<#5>{%
        % \Alt<#6>{\draw[-,dashed,ultra thick,red] ($(objects.north east)+(0.5,0.3)$) -- ($(objects.south east)+(0.5,-0.3)$);}
        % {\draw[Green,<-,thick] ($(objects.east)+(0,-0.2)$) -- ($(tracks.west)+(0,-0.2)$);
        %   \draw[-,dashed,ultra thick,Green] ($(objects.north east)+(0.5,0.3)$) -- ($(objects.south east)+(0.5,-0.3)$);
        % }
        \Alt<#7>{\draw[-,dashed,ultra thick,red] ($(tracks.north east)+(0.5,0.3)$) -- ($(tracks.south east)+(0.5,-0.3)$);}
        {\draw[Green,<-,thick] ($(tracks.east)+(0,-0.2)$) -- ($(events.west)+(0,-0.2)$);
          % \draw[-,dashed,ultra thick,Green] ($(objects.north east)+(0.5,0.3)$) -- ($(objects.south east)+(0.5,-0.3)$);
        }
        \Alt<#8>{\draw[-,dashed,ultra thick,red] ($(events.north east)+(0.5,0.3)$) -- ($(events.south east)+(0.5,-0.3)$);}
        {\draw[Green,<-,thick] ($(events.east)+(0,-0.2)$) -- ($(sentences.west)+(0,-0.2)$);
          % \draw[-,dashed,ultra thick,Green] ($(objects.north east)+(0.5,0.3)$) -- ($(objects.south east)+(0.5,-0.3)$);
        }}
    \end{tikzpicture}
  \end{figure}
}

\newcommand{\randomDetections}{%
  \draw[black,thick] (0.60,0.41) rectangle (0.69,0.53) node () {};
  \draw[black,thick] (0.06,0.08) rectangle (0.15,0.20) node () {};
  \draw[black,thick] (0.08,0.59) rectangle (0.17,0.71) node () {};
  \draw[black,thick] (0.55,0.02) rectangle (0.64,0.14) node () {};
  \draw[black,thick] (0.56,0.11) rectangle (0.65,0.23) node () {};
  \draw[black,thick] (0.03,0.72) rectangle (0.12,0.84) node () {};
  \draw[black,thick] (0.31,0.20) rectangle (0.40,0.32) node () {};
  \draw[black,thick] (0.33,0.41) rectangle (0.42,0.53) node () {};
  \draw[black,thick] (0.64,0.56) rectangle (0.73,0.68) node () {};
  \draw[black,thick] (0.31,0.56) rectangle (0.40,0.68) node () {};
  \draw[black,thick] (0.17,0.12) rectangle (0.26,0.24) node () {};
  \draw[black,thick] (0.13,0.32) rectangle (0.22,0.44) node () {};
  \draw[black,thick] (0.36,0.31) rectangle (0.45,0.43) node () {};
  \draw[black,thick] (0.63,0.40) rectangle (0.72,0.52) node () {};
  \draw[black,thick] (0.40,0.19) rectangle (0.49,0.31) node () {};
  \draw[black,thick] (0.17,0.61) rectangle (0.26,0.73) node () {};
  \draw[black,thick] (0.64,0.73) rectangle (0.73,0.85) node () {};
  \draw[black,thick] (0.50,0.64) rectangle (0.59,0.76) node () {};
  \draw[black,thick] (0.11,0.06) rectangle (0.20,0.18) node () {};
  \draw[black,thick] (0.64,0.42) rectangle (0.73,0.54) node () {};
  \draw[black,thick] (0.11,0.08) rectangle (0.20,0.20) node () {};
  \draw[black,thick] (0.36,0.76) rectangle (0.45,0.88) node () {};
  \draw[black,thick] (0.85,0.81) rectangle (0.94,0.93) node () {};
  \draw[black,thick] (0.64,0.49) rectangle (0.73,0.61) node () {};
  \draw[black,thick] (0.71,0.27) rectangle (0.80,0.39) node () {};
  \draw[black,thick] (0.59,0.32) rectangle (0.68,0.44) node () {};
  \draw[black,thick] (0.37,0.23) rectangle (0.46,0.35) node () {};
  \draw[black,thick] (0.72,0.47) rectangle (0.81,0.59) node () {};
  \draw[black,thick] (0.59,0.25) rectangle (0.68,0.37) node () {};
  \draw[black,thick] (0.08,0.55) rectangle (0.17,0.67) node () {};
  \draw[black,thick] (0.51,0.62) rectangle (0.60,0.74) node () {};
  \draw[black,thick] (0.30,0.78) rectangle (0.39,0.90) node () {};
  \draw[black,thick] (0.61,0.34) rectangle (0.70,0.46) node () {};
  \draw[black,thick] (0.23,0.70) rectangle (0.32,0.82) node () {};
  \draw[black,thick] (0.66,0.26) rectangle (0.75,0.38) node () {};
  \draw[black,thick] (0.12,0.81) rectangle (0.21,0.93) node () {};
  \draw[black,thick] (0.48,0.46) rectangle (0.57,0.58) node () {};
  \draw[black,thick] (0.23,0.34) rectangle (0.32,0.46) node () {};
  \draw[black,thick] (0.40,0.21) rectangle (0.49,0.33) node () {};
  \draw[black,thick] (0.69,0.60) rectangle (0.78,0.72) node () {};
  \draw[black,thick] (0.08,0.28) rectangle (0.17,0.40) node () {};
  \draw[black,thick] (0.51,0.24) rectangle (0.60,0.36) node () {};
  \draw[black,thick] (0.63,0.55) rectangle (0.72,0.67) node () {};
  \draw[black,thick] (0.23,0.46) rectangle (0.32,0.58) node () {};
  \draw[black,thick] (0.41,0.26) rectangle (0.50,0.38) node () {};
  \draw[black,thick] (0.56,0.03) rectangle (0.65,0.15) node () {};
  \draw[black,thick] (0.57,0.56) rectangle (0.66,0.68) node () {};
  \draw[black,thick] (0.81,0.24) rectangle (0.90,0.36) node () {};
  \draw[black,thick] (0.69,0.20) rectangle (0.78,0.32) node () {};
  \draw[black,thick] (0.14,0.63) rectangle (0.23,0.75) node () {};
  \draw[black,thick] (0.57,0.18) rectangle (0.66,0.30) node () {};
  \draw[black,thick] (0.49,0.82) rectangle (0.58,0.94) node () {};
  \draw[black,thick] (0.14,0.31) rectangle (0.23,0.43) node () {};
  \draw[black,thick] (0.47,0.35) rectangle (0.56,0.47) node () {};
  \draw[black,thick] (0.71,0.10) rectangle (0.80,0.22) node () {};
  \draw[black,thick] (0.13,0.13) rectangle (0.22,0.25) node () {};
  \draw[black,thick] (0.82,0.83) rectangle (0.91,0.95) node () {};
  \draw[black,thick] (0.57,0.18) rectangle (0.66,0.30) node () {};
  \draw[black,thick] (0.56,0.42) rectangle (0.65,0.54) node () {};
  \draw[black,thick] (0.36,0.07) rectangle (0.45,0.19) node () {};
  \draw[black,thick] (0.09,0.58) rectangle (0.18,0.70) node () {};
  \draw[black,thick] (0.52,0.81) rectangle (0.61,0.93) node () {};
  \draw[black,thick] (0.77,0.47) rectangle (0.86,0.59) node () {};
  \draw[black,thick] (0.50,0.06) rectangle (0.59,0.18) node () {};
  \draw[black,thick] (0.14,0.09) rectangle (0.23,0.21) node () {};
  \draw[black,thick] (0.25,0.56) rectangle (0.34,0.68) node () {};
  \draw[black,thick] (0.68,0.50) rectangle (0.77,0.62) node () {};
  \draw[black,thick] (0.19,0.73) rectangle (0.28,0.85) node () {};
  \draw[black,thick] (0.58,0.30) rectangle (0.67,0.42) node () {};
  \draw[black,thick] (0.85,0.14) rectangle (0.94,0.26) node () {};
  \draw[black,thick] (0.83,0.48) rectangle (0.92,0.60) node () {};
  \draw[black,thick] (0.40,0.41) rectangle (0.49,0.53) node () {};
  \draw[black,thick] (0.18,0.58) rectangle (0.27,0.70) node () {};
  \draw[black,thick] (0.08,0.23) rectangle (0.17,0.35) node () {};
  \draw[black,thick] (0.51,0.15) rectangle (0.60,0.27) node () {};
  \draw[black,thick] (0.08,0.58) rectangle (0.17,0.70) node () {};
}

\tikzset{RPY/.style={x={(\nxx,\nxy)},y={(\nyx,\nyy)},z={(\nzx,\nzy)}}}

\usepackage[outline]{contour}

\makeatletter
\tikzset{%
  fitting node/.style={%
    inner sep=0pt,
    fill=none,
    draw=none,
    reset transform,
    fit={(\pgf@pathminx,\pgf@pathminy) (\pgf@pathmaxx,\pgf@pathmaxy)}
  },
  reset transform/.code={\pgftransformreset}
}
\makeatother

% \makeatletter

% \newcommand*\Alt{\alt{\Alt@branch0}{\Alt@branch1}}

% \newcommand\Alt@branch[3]{%
%   \begingroup
%   \ifbool{mmode}{%
%     \mathchoice{%
%       \Alt@math#1{\displaystyle}{#2}{#3}%
%     }{%
%       \Alt@math#1{\textstyle}{#2}{#3}%
%     }{%
%       \Alt@math#1{\scriptstyle}{#2}{#3}%
%     }{%
%       \Alt@math#1{\scriptscriptstyle}{#2}{#3}%
%     }%
%   }{%
%     \sbox0{#2}%
%     \sbox1{#3}%
%     \Alt@typeset#1%
%   }%
%   \endgroup
% }

% \newcommand\Alt@math[4]{%
%   \sbox0{$#2{#3}\m@th$}%
%   \sbox1{$#2{#4}\m@th$}%
%   \Alt@typeset#1%
% }

% \newcommand\Alt@typeset[1]{%
%   \ifnumcomp{\wd0}{>}{\wd1}{%
%     \def\setwider   ##1##2{##2##1##2 0}%
%     \def\setnarrower##1##2{##2##1##2 1}%
%   }{%
%     \def\setwider   ##1##2{##2##1##2 1}%
%     \def\setnarrower##1##2{##2##1##2 0}%
%   }%
%   \sbox2{}%
%   \sbox3{}%
%   \setwider2{\wd}%
%   \setwider2{\ht}%
%   \setwider2{\dp}%
%   \setnarrower3{\ht}%
%   \setnarrower3{\dp}%
%   \leavevmode
%   \rlap{\usebox#1}%
%   \usebox2%
%   \usebox3%
% }
% \makeatother

\makeatletter

\newcommand<>\Alt[3][l]{%
  \begingroup%
  \providetoggle{Alt@before}%
  \alt#4{\toggletrue{Alt@before}}{\togglefalse{Alt@before}}%
  \ifbool{mmode}{%
    \expandafter\mathpalette%
    \expandafter\math@Alt%
  }{%
    \expandafter\make@Alt%
  }%
  {{#1}{#2}{#3}}%
  \endgroup%
}

% Un-brace the second argument (required because \mathpalette reads the three arguments as one)
\newcommand\math@Alt[2]{\math@@Alt{#1}#2}

% Set the two arguments in boxes. The math style is given by #1. \m@th sets \mathsurround to 0.
\newcommand\math@@Alt[4]{%
  \setbox\z@ \hbox{$\m@th #1{#3}$}%
  \setbox\@ne\hbox{$\m@th #1{#4}$}%
  \@Alt{#2}%
}

% Un-brace the argument
\newcommand\make@Alt[1]{\make@@Alt#1}

% Set the two arguments into normal boxes
\newcommand\make@@Alt[3]{%
  \sbox\z@ {#2}%
  \sbox\@ne{#3}%
  \@Alt{#1}%
}

% Place one of the two boxes using \rlap and place a \phantom box with the maximum of the two boxes
\newcommand\@Alt[1]{%
  \setbox\tw@\null
  \ht\tw@\ifnum\ht\z@>\ht\@ne\ht\z@\else\ht\@ne\fi
  \dp\tw@\ifnum\dp\z@>\dp\@ne\dp\z@\else\dp\@ne\fi
  \wd\tw@\ifnum\wd\z@>\wd\@ne\dimexpr\wd\z@/2\relax\else\dimexpr\wd\@ne/2\relax\fi
  %
  \ifstrequal{#1}{l}{%
    \rlap{\iftoggle{Alt@before}{\usebox\z@}{\usebox\@ne}}%
    \copy\tw@
    \box\tw@
  }{%
    \ifstrequal{#1}{c}{%
      \copy\tw@
      \clap{\iftoggle{Alt@before}{\usebox\z@}{\usebox\@ne}}%
      \box\tw@
    }{%
      \ifstrequal{#1}{r}{%
        \copy\tw@
        \box\tw@
        \llap{\iftoggle{Alt@before}{\usebox\z@}{\usebox\@ne}}%
      }{%
      }%
    }%
  }%
}

\makeatother

\newcommand<>{\bgOn}[2]{\temporal#3{\colorbox{bg}{#2}}{\colorbox{#1}{#2}}{\colorbox{bg}{#2}}}

\ifdefined\darkmode
% Dark
\definecolor{Red}{rgb}{1,0,0}
\definecolor{Green}{rgb}{0,0.69,0}
\definecolor{Blue}{rgb}{0,0,1}
\definecolor{LightBlue}{rgb}{0,0.5,1}
\definecolor{LightRed}{rgb}{1,0.2,0.2}
\definecolor{VeryLightRed}{rgb}{1,0.4,0.4}
\definecolor{ExtremelyLightRed}{rgb}{1,0.6,0.6}
\definecolor{Skin}{rgb}{1,0.71,0.69}
\definecolor{Grey}{rgb}{0.5,0.5,0.5}
\definecolor{LightGrey}{rgb}{0.6,0.6,0.6}
\definecolor{Black}{rgb}{0,0,0}
\definecolor{White}{rgb}{1,1,1}
\definecolor{MitRed}{rgb}{0.6,0.2,0.2}
\newcommand{\red}{\color{Red}}
\newcommand{\mitred}{\color{MitRed}}
\newcommand{\green}{\color{Green}}
\newcommand{\blue}{\color{Blue}}
\newcommand{\lightBlue}{\color{LightBlue}}
\newcommand{\lightRed}{\color{LightRed}}
\newcommand{\veryLightRed}{\color{VeryLightRed}}
\newcommand{\extremelyLightRed}{\color{ExtremelyLightRed}}
\newcommand{\skin}{\color{Skin}}
\newcommand{\redOn}{\only{\red}}
\newcommand{\lightRedOn}{\only{\lightRed}}
\newcommand{\veryLightRedOn}{\only{\veryLightRed}}
\newcommand{\extremelyLightRedOn}{\only{\extremelyLightRed}}
\newcommand{\greenOn}{\only{\green}}
\newcommand{\blueOn}{\only{\blue}}
\newcommand{\lightBlueOn}{\only{\lightBlue}}
\newcommand{\skinOn}{\only{\skin}}
\newcommand{\gold}{\color{i6colorscheme6}}
\newcommand{\goldOn}{\only{\gold}}
\newcommand{\grey}{\color{Grey}}
\newcommand{\greyOn}{\only{\grey}}
\newcommand{\lightGrey}{\color{LightGrey}}
\newcommand{\lightGreyOn}{\only{\lightGrey}}
\newcommand{\black}{\color{Black}}
\newcommand{\blackOn}{\only{\black}}
\newcommand{\orange}{\color{Bittersweet}}
\newcommand{\orangeOn}{\only{\orange}}
\newcommand{\violet}{\color{BlueViolet}}
\newcommand{\violetOn}{\only{\violet}}
\newcommand{\white}{\color{White}}
\newcommand{\whiteOn}{\only{\white}}
\else
% Light
\definecolor{Red}{rgb}{1,0.05,0.05}
\definecolor{Green}{rgb}{0,0.8,0}
\definecolor{Blue}{rgb}{0,0,1}
\definecolor{LightBlue}{rgb}{0,0.5,1}
\definecolor{LightRed}{rgb}{1,0.25,0.25}
\definecolor{VeryLightRed}{rgb}{1,0.4,0.4}
\definecolor{ExtremelyLightRed}{rgb}{1,0.6,0.6}
\definecolor{Skin}{rgb}{1,0.71,0.69}
\definecolor{Grey}{rgb}{0.8,0.8,0.8}
\definecolor{LightGrey}{rgb}{0.6,0.6,0.6}
\definecolor{Black}{rgb}{0,0,0}
\definecolor{White}{rgb}{1,1,1}
\definecolor{MitRed}{rgb}{0.6,0.2,0.2}
\newcommand{\red}{\color{Red}}
\newcommand{\mitred}{\color{MitRed}}
\newcommand{\green}{\color{Green}}
\newcommand{\blue}{\color{Blue}}
\newcommand{\lightBlue}{\color{LightBlue}}
\newcommand{\lightRed}{\color{LightRed}}
\newcommand{\veryLightRed}{\color{VeryLightRed}}
\newcommand{\extremelyLightRed}{\color{ExtremelyLightRed}}
\newcommand{\skin}{\color{Skin}}
\newcommand{\redOn}{\only{\red}}
\newcommand{\lightRedOn}{\only{\lightRed}}
\newcommand{\veryLightRedOn}{\only{\veryLightRed}}
\newcommand{\extremelyLightRedOn}{\only{\extremelyLightRed}}
\newcommand{\greenOn}{\only{\green}}
\newcommand{\blueOn}{\only{\blue}}
\newcommand{\lightBlueOn}{\only{\lightBlue}}
\newcommand{\skinOn}{\only{\skin}}
\newcommand{\gold}{\color{i6colorscheme6}}
\newcommand{\goldOn}{\only{\gold}}
\newcommand{\grey}{\color{Grey}}
\newcommand{\greyOn}{\only{\grey}}
\newcommand{\gray}{\color{Grey}}
\newcommand{\grayOn}{\only{\grey}}
\newcommand{\lightGrey}{\color{LightGrey}}
\newcommand{\lightGreyOn}{\only{\lightGrey}}
\newcommand{\black}{\color{Black}}
\newcommand{\blackOn}{\only{\black}}
\newcommand{\orange}{\color{Apricot}}
\newcommand{\orangeOn}{\only{\orange}}
\newcommand{\violet}{\color{BlueViolet}}
\newcommand{\violetOn}{\only{\violet}}
\newcommand{\white}{\color{White}}
\newcommand{\whiteOn}{\only{\white}}
\fi

% Dark
% \definecolor{Red}{rgb}{1,0,0}
% \definecolor{LightRed}{rgb}{0.8,0.2,0.2}
% \definecolor{Green}{rgb}{0,0.69,0}
% \definecolor{Blue}{rgb}{0,0,1}
% \definecolor{LightBlue}{rgb}{0,0.5,1}
% \definecolor{lightBlue}{rgb}{0,0.5,1}
% \definecolor{Skin}{rgb}{1,0.71,0.69}
% \definecolor{Grey}{rgb}{0.5,0.5,0.5}
% \definecolor{LightGrey}{rgb}{0.6,0.6,0.6}
% \definecolor{Black}{rgb}{0,0,0}
% \definecolor{White}{rgb}{1,1,1}
% \newcommand{\red}{\color{Red}}
% \newcommand{\green}{\color{Green}}
% \newcommand{\blue}{\color{Blue}}
% \newcommand{\lightBlue}{\color{LightBlue}}
% \newcommand{\lightRed}{\color{LightRed}}
% \newcommand{\veryLightRed}{\color{VeryLightRed}}
% \newcommand{\extremelyLightRed}{\color{ExtremelyLightRed}}
% \newcommand{\skin}{\color{Skin}}
% \newcommand{\redOn}{\only{\red}}
% \newcommand{\lightRedOn}{\only{\lightRed}}
% \newcommand{\veryLightRedOn}{\only{\veryLightRed}}
% \newcommand{\extremelyLightRedOn}{\only{\extremelyLightRed}}
% \newcommand{\greenOn}{\only{\green}}
% \newcommand{\blueOn}{\only{\blue}}
% \newcommand{\lightBlueOn}{\only{\lightBlue}}
% \newcommand{\skinOn}{\only{\skin}}
% \newcommand{\gold}{\color{i6colorscheme6}}
% \newcommand{\goldOn}{\only{\gold}}
% \newcommand{\grey}{\color{Grey}}
% \newcommand{\greyOn}{\only{\grey}}
% \newcommand{\lightGrey}{\color{LightGrey}}
% \newcommand{\lightGreyOn}{\only{\lightGrey}}
% \newcommand{\black}{\color{Black}}
% \newcommand{\blackOn}{\only{\black}}
% \newcommand{\orange}{\color{Bittersweet}}
% \newcommand{\orangeOn}{\only{\orange}}
% \newcommand{\violet}{\color{BlueViolet}}
% \newcommand{\violetOn}{\only{\violet}}
% \newcommand{\white}{\color{White}}
% \newcommand{\whiteOn}{\only{\white}}

% Dark colors if I ever want to revive them again
%

% from http://tex.stackexchange.com/questions/9559/drawing-on-an-image-with-tikz
\newcommand{\tikzgrid}{%
  \draw[help lines,xstep=.1,ystep=.1] (0,0) grid (1,1);
  \foreach \x in {0,1,...,9} { \node [anchor=north] at (\x/10,0) {0.\x}; }
  \foreach \y in {0,1,...,9} { \node [anchor=east] at (0,\y/10) {0.\y}; }}
\newcommand{\tikzgridfour}{%
  \draw[help lines,xstep=1,ystep=1] (0,0) grid (4,4);
  \foreach \x in {0,1,...,4} { \node [anchor=north] at (\x,0) {\x}; }
  \foreach \y in {0,1,...,4} { \node [anchor=east] at (0,\y) {\y}; }}

\usepackage{pgfplots}
\usepackage{helvet}
\usepackage[eulergreek]{sansmath}
\usepackage{stringstrings}
\usetikzlibrary{calc}
\usetikzlibrary{fit}
\usetikzlibrary{arrows}
\usetikzlibrary{positioning}
\usetikzlibrary{decorations.markings}
\usetikzlibrary{shapes.geometric}
\usetikzlibrary{shapes.multipart}

\pgfplotsset{
  tick label style = {font=\sansmath\sffamily}
}

\def\stateSize{0.4}
\def\stateHeight{4}
\def\stateArcHeight{0.6}
\def\rectWidth{0.3}
\def\rectHeight{0.15}
\def\rectOffset{-1.2}
\def\featNamePos{-0.7}
\def\stateSep{1.5}

\newcommand{\mystate}[5]{
  \draw[] ($(#1,\stateHeight)+(0,#3)$) circle (\stateSize)
  node[fitting node] (#2) {};
  %% cannot directly compare floating numbers
  \newdimen\dummy
  \dummy = #4 pt
  \ifthenelse{\dummy > 0 pt}{
    \node[] (p) at ($(#2.north)+(0,\stateArcHeight)$) {#4};
    \draw [->, thick] (#2.north) to[out=15,in=0] ($(p.south)-(0,0.05)$)
    to[out=180,in=165] (#2.north);
  }{}
  \dummy = #5 pt
  \ifthenelse{\dummy > 0 pt}{
    \draw [->, thick] ($(#2.south west)+(-0.3,-0.1)$) -- (#2) node [midway, above,
      sloped] {\scalebox{0.7}{#5}};
  }{}
}

\makeatletter
\tikzset{
  fitting node/.style={
    inner sep=0pt,
    fill=none,
    draw=none,
    reset transform,
    fit={(\pgf@pathminx,\pgf@pathminy) (\pgf@pathmaxx,\pgf@pathmaxy)}
  },
  reset transform/.code={\pgftransformreset}
}
\makeatother

\newcommand{\tikzfig}[2]{
  \begin{tikzpicture}[#1] #2 \end{tikzpicture}}

\newcommand{\tscope}[2]{\begin{scope}[x={(#1.south east)},y={(#1.north west)}]
    #2 \end{scope}}

\newcommand{\insertImage}[3]{
  \node[anchor=south west,inner sep=0] (#3) at (0,0)
       {\includegraphics[width=#2\textwidth]
         {#1}};}

\newcommand{\variable}[4]{
    \draw[thick] (#1,#2) circle (\stateSize) node[fitting node] (#3) {};
    \node[#4] (name-#3) at ([yshift=-16] #3) {\textbf{#3}};
}

\newcommand{\constraint}[6]{
  \node (c) at ($(#1,#2)$)
        {\includegraphics[width=0.4\textwidth]{#6}};
  \draw[thick] (#3) -- (c);
  \draw[thick] (#4) -- (c);
  \draw[thick] (#5) -- (c);
}

% #1-#5 node lable #6 node name #7,#8 x,y
\newcommand{\latticeNodeFive}[8]{
  \node[draw, minimum width=3cm, minimum height=1cm, anchor=south] (#6) at (#7, #8)
       {$\substack{#4,#5\\#1,#2,#3}$};
}

% #1-#3 node lable #4 node name #5,#6 x,y
\newcommand{\latticeNodeThree}[6]{
  \node[draw, minimum width=2.4cm, minimum height=0.85cm, anchor=south] (#4) at (#5, #6)
       {#1, #2, #3};
}

% #1,#2 node lable #3 node name #4,#5 x,y
\newcommand{\latticeNodeTwo}[5]{
  \node[draw, minimum width=1.7cm, minimum height=0.85cm, anchor=south] (#3) at (#4, #5)
       {#1, #2};
}

% #1 node lable #2 node name #3,#4 x,y #5 draw border
\newcommand{\latticeNode}[5]{
  \node[#5,minimum width=1.7cm, minimum height=1.06cm, anchor=south] (#2) at (#3,#4) {#1};
}

% the two types of lattices are scaled differently, so need two functions
% #1 supernode label, #2 node block colour #3 node block label
% #4 link block colour #5 link block label, #6 lattice label
\newcommand{\latticeOverlayT}[6]{
  \fill[#2,opacity=0.25]
  let
  \p1=($(#1.west)!0.125!(#1.east)$),
  \p2=($(#1.west)!0.235!(#1.east)$),
  \p3=(#1.south),
  \p4=($(#1.north)!0.08!(#1.south)$)
  in (\x1,\y3) rectangle (\x2, \y4);
  \fill[#4,opacity=0.25]
  let
  \p1=($(#1.west)!0.235!(#1.east)$),
  \p2=($(#1.west)!0.343!(#1.east)$),
  \p3=($(#1.south)!0.057!(#1.north)$),
  \p4=($(#1.north)!0.138!(#1.south)$)
  in (\x1,\y3) rectangle (\x2, \y4);
  \path
  let
  \p1=($(#1.west)!0.1795!(#1.east)$),
  \p2=($(#1.south)$) in node[#2,anchor=south] at ($(\x1,\y2)-(0,0.035)$) {#3};
  \path let \p1=($(#1.west)!0.289!(#1.east)$), \p2=($(#1.south)$) in
  node[#4,anchor=south] at ($(\x1,\y2)-(0,0.035)$) {#5};
  \path let \p1=($(#1.west)!0.5!(#1.east)$), \p2=($(#1.south)$) in node (#1l) at
  ($(\x1,\y2)-(0,0.05)$) {#6};
}

\newcommand{\latticeOverlayW}[6]{
  \fill[#2,opacity=0.25]
  let
  \p1=($(#1.west)!0.177!(#1.east)$),
  \p2=($(#1.west)!0.291!(#1.east)$),
  \p3=(#1.south),
  \p4=($(#1.north)!0.11!(#1.south)$)
  in (\x1,\y3) rectangle (\x2, \y4);
  \fill[#4,opacity=0.25]
  let
  \p1=($(#1.west)!0.291!(#1.east)$),
  \p2=($(#1.west)!0.379!(#1.east)$),
  \p3=($(#1.south)!0.068!(#1.north)$),
  \p4=($(#1.north)!0.171!(#1.south)$)
  in (\x1,\y3) rectangle (\x2, \y4);
  \path
  let
  \p1=($(#1.west)!0.234!(#1.east)$),
  \p2=($(#1.south)$) in node[#2,anchor=south] at ($(\x1,\y2)-(0,0.035)$) {#3};
  \path let \p1=($(#1.west)!0.335!(#1.east)$), \p2=($(#1.south)$) in
  node[#4,anchor=south] at ($(\x1,\y2)-(0,0.035)$) {#5};
  \path let \p1=($(#1.west)!0.5!(#1.east)$), \p2=($(#1.south)$) in node (#1l) at
  ($(\x1,\y2)-(0,0.05)$) {#6};
}

\tikzstyle{vecArrow} = [thick, decoration={markings,mark=at position
   1 with {\arrow[semithick]{open triangle 60}}},
   double distance=3pt, shorten >= 5.5pt,
   preaction = {decorate},
   postaction = {draw,line width=1.4pt, white,shorten >= 4.5pt}]
\tikzstyle{innerWhite} = [semithick, white,line width=1.4pt, shorten >= 4.5pt]


\renewcommand<>{\sout}[1]{\alt#2{\beameroriginal{\sout}{#1}}{#1}}

%%%%%%%%%%%%%%%%%%%%%%%%%%%%%%%%%%%%%%%%%%%%%%%%%%%%%%%%%%%%%%%%%%%%%%%%%%%%%%%
% Madrid is default

% \usetheme{CambridgeUS}
% \usetheme{cbmm}
% \usetheme{Madrid}
% \usetheme{default}
\useoutertheme{cbmm}
% \usefonttheme{serif}

% % I like the the other theme more
% % \usetheme{CambridgeUS}
\usetheme{Madrid}
% \useoutertheme{slidenum}
% \useoutertheme{noslidenum}
% \usefonttheme{serif}

\newcommand{\bgbox}[1]{\colorbox{white}{#1}}
% \usecolortheme{purdue}
% \usecolortheme{cbmm}
% \usecolortheme{mule}

\ifdefined\darkmode
% For dark colors
\usecolortheme{mule}
\else
\usecolortheme{mit}
\fi

\makeatletter
\NewDocumentCommand{\sotwo}{O{red}O{black}+m}
{%
  \begingroup
  \setulcolor{#1}%
  \setul{-.5ex}{.4pt}%
  \def\SOUL@uleverysyllable{%
    \rlap{%
      \color{#2}\the\SOUL@syllable
      \SOUL@setkern\SOUL@charkern}%
    \SOUL@ulunderline{%
      \phantom{\the\SOUL@syllable}}%
  }%
  \ul{#3}%
  \endgroup
}
\makeatother

\graphicspath{{game-learning/videos}{game-learning/}{ll/images}{ll/}{images/}{videos/}{figures/}{people/}{images/objectnet-2019}{/home/andrei/Dropbox/work/cbmm-talk/icra2019/images/}}
\DeclareGraphicsExtensions{.jpg,.png,.pdf}

\newcommand\redsout{\bgroup\markoverwith{\textcolor{red}{\rule[0.5ex]{2pt}{0.6pt}}}\ULon}

\tikzset{
    between/.style args={#1 and #2}{
         at = ($(#1)!0.5!(#2)$)
    }
}
\tikzset{
    between base/.style args={#1 and #2}{
        between=#1.base and #2.base
    }
}

\newcommand<>{\uncoverubrace}[2]{%
  \onslide#3 \underbrace{ \onslide<1->%
  #1%
  \onslide#3 }_{#2} \onslide<1->%
}
\newcommand<>{\uncoverobrace}[2]{%
  \onslide#3 \overbrace{ \onslide<1->%
  #1%
  \onslide#3 }^{#2} \onslide<1->%
}

\newcommand<>{\hl}[2]{%
  {\Alt#3{\mathchoice%
  {\colorbox{#1}{$\displaystyle#2$}}%
  {\colorbox{#1}{$\textstyle#2$}}%
  {\colorbox{#1}{$\scriptstyle#2$}}%
  {\colorbox{#1}{$\scriptscriptstyle#2$}}}{#2}}}

\newcommand<>{\hlnarrow}[2]{%
  \setlength\fboxsep{0pt}%
  {\Alt#3{\mathchoice%
  {\colorbox{#1}{$\displaystyle#2$}}%
  {\colorbox{#1}{$\textstyle#2$}}%
  {\colorbox{#1}{$\scriptstyle#2$}}%
  {\colorbox{#1}{$\scriptscriptstyle#2$}}}{#2}}}

\usepackage{cmbright}

\setbeamertemplate{frametitle}
{
  \vspace{1ex}
  \contourlength{0.05pt}
  \contournumber{5}
  \contour{mitred}{\insertframetitle}
  \vspace{1ex}
  % \insertframetitle
}

\makeatletter
\DeclareRobustCommand*{\savepause}[1]{\only<1>{\immediate\write\@auxout{\string\pauseentry{\the\c@framenumber}{#1}{\the\c@beamerpauses}}}}
\newcommand*{\usepause}[1]{\@ifundefined{pauses@\the\c@framenumber @#1}{1}{\@nameuse{pauses@\the\c@framenumber @#1}}}
\newcommand*{\pauseentry}[3]{\global\@namedef{pauses@#1@#2}{#3}}
\makeatother


\def\ltlX{\mathbin{\scalerel*{\bigcirc}{\forall}\hspace{0.1ex}}}
\def\ltlnext{\ltlX}
\def\ltlF{\mathbin{\scalerel*{\square}{\forall}\hspace{0.1ex}}}
\def\ltlfinally{\ltlF}
\def\ltlG{\mathbin{\scalerel*{\lozenge}{\forall}\hspace{0.1ex}}}
\def\ltlglobally{\ltlG}
